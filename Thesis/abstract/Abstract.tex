\begin{abstract}
Topographic maps are ubiquitous brain structures that are fundamental to sensory and higher order systems and are composed of connections between two regions obeying the relationship: physically neighbouring cells in a pre-synaptic region connect to physically neighbouring cells in the post-synaptic region. The developmental principles driving topographic map formation are usually studied within the context of genetic perturbations coupled to high resolution measurements and for these the mouse retinotopic map from retina to superior colliculus has emerged as a useful experimental context. Modelling coupled with genetic perturbation experiments has revealed three key developmental mechanisms: neural activity, chemotaxis, and competition. Some principal challenges in modelling this development include explaining the role of the spatio-temporal structure of patterned neural activity, determining the relative interaction between developmental components, and developing models that are sufficiently computationally efficient that statistical methodologies can be applied to them.
\\\\
Neural activity is a well measured component of retinotopic development and several independent measurement techniques have recorded the existence of spatiotemporally patterned waves at key critical points during development. Existing modelling methodologies reduce this rich spatiotemporal context into a distance dependent correlation function and have subsequently had challenges making quantitative predictions about the effect of manipulating these activity patterns. A neural field theory approach is used to develop mathematical theory which can incorporate these spatiotemporal structures. Bayesian MCMC regression analysis is performed on biological measurements to assess the accuracy of the model and make predictions about the time-scale on which activity operates. This time scale is tuned to the length of an average wave pattern suggesting the system is integrating all information in these waves.
\\\\
The interaction between chemotaxis and neural activity has historically been thought of as linearly independent. A recent study which perturbs both developmental mechanisms simultaneously has suggested that these two are highly stochastic and regular development depends on a critical fine-tuned balance between the two: the heterozygous phenotype was observed to present as both a wild-type and homozygote for different specimens. This hypothesis is tested against the data-set used to generate it. Recreating the entire experimental pipeline \textit{in silico} with the most parsimonious existing model is able to account for the data without the need to appeal to stochasticity in the mechanisms. A statistical analysis demonstrates that the heterozygous state does not significantly overlap with the heterozygotes and that the stochasticity is likely due to the measurement technique. 
\\\\
The existing models are computationally demanding; at least $O(n^4)$ in the number of retinal cells instantiated by the model. This computational demand renders these classes of models incapable of performing statistical regression and means that their parameters spaces are largely unexplored. A modelling framework which integrates the core operating mechanisms of the model is developed and when implemented on modern GPU computational architectures is able to achieve a sub-linear time complexity scaling. This model is demonstrated to capture the explanatory power of existing modelling methodologies.
\\\\
Finally, the role of competition is explored in a dimensional reduction framework: the Elastic Net. The Elastic Net has been used both as a heuristic optimiser (validated on the NP-complete Travelling Salesman Problem) and to explain the development of cortical feature maps. The addition of competition is demonstrated to act as a counter-measure to the retinotopic distorting components of the Elastic Net as a cortical map generator. Further analysis demonstrates that competition substantially improves heuristic performance on the Travelling Salesman Problem making it competitive against state of the art solvers when solution times are normalised. The heuristic converges on a length scaling law that is discussed in the context of wire-minimisation problem. 
\end{abstract}
